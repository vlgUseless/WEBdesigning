\documentclass[a4paper, final]{article}
%\usepackage{literat} % Нормальные шрифты
\usepackage[14pt]{extsizes} % для того чтобы задать нестандартный 14-ый размер шрифта
\usepackage[T2A]{fontenc}
\usepackage[UTF8]{inputenc}
\usepackage[russian]{babel}
\usepackage{listings} %листинги
\usepackage{amsmath}
\usepackage{amssymb} % Для красивого значка пустого множества
\usepackage[left=25mm, top=20mm, right=20mm, bottom=20mm, footskip=10mm]{geometry}
\usepackage{ragged2e} %для растягивания по ширине
\usepackage{setspace} %для межстрочного интервала
\usepackage{indentfirst} % для абзацного отступа
\usepackage{moreverb} %для печати в листинге исходного кода программ
\renewcommand\verbatimtabsize{4\relax}
\renewcommand\listingoffset{0.2em} %отступ от номеров строк в листинге
\renewcommand{\arraystretch}{1.4} % изменяю высоту строки в таблице
\usepackage[font=small, singlelinecheck=false, justification=centering, format=plain, labelsep=period]{caption} %для настройки заголовка таблицы
\usepackage{listingsutf8}
\usepackage{xcolor} % цвета
\usepackage{hyperref}% для гиперссылок
\usepackage{enumitem} %для перечислений
\usepackage{titlesec}
\usepackage{graphicx}
\graphicspath{ {./Рисунки/} }
%\usepackage{float}
\usepackage{booktabs}
\usepackage{floatrow}
\usepackage{scalerel} % Stretching images
\usepackage[final]{pdfpages}

\definecolor{apricot}{HTML}{FFF0DA}
\definecolor{mygreen}{rgb}{0,0.6,0}
\definecolor{string}{HTML}{B40000} % цвет строк в коде
\definecolor{comment}{HTML}{008000} % цвет комментариев в коде
\definecolor{keyword}{HTML}{1A00FF} % цвет ключевых слов в коде
\definecolor{morecomment}{HTML}{8000FF} % цвет include и других элементов в коде
\definecolor{captiontext}{HTML}{FFFFFF} % цвет текста заголовка в коде
\definecolor{captionbk}{HTML}{999999} % цвет фона заголовка в коде
\definecolor{bk}{HTML}{FFFFFF} % цвет фона в коде
\definecolor{frame}{HTML}{999999} % цвет рамки в коде
\definecolor{brackets}{HTML}{B40000} % цвет скобок в коде





\AtBeginDocument{\renewcommand{\contentsname}{Содержание}}
\AtBeginDocument{\renewcommand{\refname}{Список источников}}

\floatsetup[table]{style=plain,capposition=bottom}
\setlist[enumerate,itemize]{leftmargin=1.2cm} %отступ в перечислениях

\hypersetup{colorlinks,
  allcolors=[RGB]{010 090 200}} %красивые гиперссылки (не красные)

% подгружаемые языки — подробнее в документации listings (это всё для листингов)
\lstloadlanguages{ [LaTeX] TeX}
% включаем кириллицу и добавляем кое−какие опции
\lstset{language =[LaTeX] TeX, % выбираем язык по умолчанию
extendedchars=true , % включаем не латиницу
escapechar = | , % |«выпадаем» в LATEX|
frame=tb , % рамка сверху и снизу
commentstyle=\itshape , % шрифт для комментариев
stringstyle =\bfseries} % шрифт для строк

\textheight=24cm % высота текста
\textwidth=16cm % ширина текста
\oddsidemargin=0pt % отступ от левого края
\topmargin=-1.5cm % отступ от верхнего края
\parindent=24pt % абзацный отступ
\parskip=0pt % интервал между абзацами
\tolerance=2000 % терпимость к "жидким" строкам
\flushbottom % выравнивание высоты страниц

\begin{document} % начало документа
\lstset{
  language=SQL, % Язык кода по умолчанию
  morekeywords={*,...}, % если хотите добавить ключевые слова, то добавляйте
  % Цвета
  keywordstyle=\color{keyword}\ttfamily\bfseries,
  %stringstyle=\color{string}\ttfamily,
  stringstyle=\ttfamily\color{red!50!brown},
  commentstyle=\color{comment}\ttfamily,
  morecomment=[l][\color{morecomment}]{\#},
  % Настройки отображения
  breaklines=true, % Перенос длинных строк
  basicstyle=\ttfamily\footnotesize, % Шрифт для отображения кода
  backgroundcolor=\color{bk}, % Цвет фона кода
  frame=single,xleftmargin=\fboxsep,xrightmargin=-\fboxsep, % Рамка, подогнанная к заголовку
  rulecolor=\color{frame}, % Цвет рамки
  tabsize=3, % Размер табуляции в пробелах
  % Настройка отображения номеров строк. Если не нужно, то удалите весь блок
  numbers=left, % Слева отображаются номера строк
  stepnumber=1, % Каждую строку нумеровать
  numbersep=5pt, % Отступ от кода
  numberstyle=\small\color{black}, % Стиль написания номеров строк
  % Для отображения русского языка
  extendedchars=true,
  literate={Ö}{ {\"O} }1
  {~}{ {\textasciitilde} }1
  {а}{ {\selectfont\char224} }1
  {б}{ {\selectfont\char225} }1
  {в}{ {\selectfont\char226} }1
  {г}{ {\selectfont\char227} }1
  {д}{ {\selectfont\char228} }1
  {е}{ {\selectfont\char229} }1
  {ё}{ {\"e} }1
  {ж}{ {\selectfont\char230} }1
  {з}{ {\selectfont\char231} }1
  {и}{ {\selectfont\char232} }1
  {й}{ {\selectfont\char233} }1
  {к}{ {\selectfont\char234} }1
  {л}{ {\selectfont\char235} }1
  {м}{ {\selectfont\char236} }1
  {н}{ {\selectfont\char237} }1
  {о}{ {\selectfont\char238} }1
  {п}{ {\selectfont\char239} }1
  {р}{ {\selectfont\char240} }1
  {с}{ {\selectfont\char241} }1
  {т}{ {\selectfont\char242} }1
  {у}{ {\selectfont\char243} }1
  {ф}{ {\selectfont\char244} }1
  {х}{ {\selectfont\char245} }1
  {ц}{ {\selectfont\char246} }1
  {ч}{ {\selectfont\char247} }1
  {ш}{ {\selectfont\char248} }1
  {щ}{ {\selectfont\char249} }1
  {ъ}{ {\selectfont\char250} }1
  {ы}{ {\selectfont\char251} }1
  {ь}{ {\selectfont\char252} }1
  {э}{ {\selectfont\char253} }1
  {ю}{ {\selectfont\char254} }1
  {я}{ {\selectfont\char255} }1
  {А}{ {\selectfont\char192} }1
  {Б}{ {\selectfont\char193} }1
  {В}{ {\selectfont\char194} }1
  {Г}{ {\selectfont\char195} }1
  {Д}{ {\selectfont\char196} }1
  {Е}{ {\selectfont\char197} }1
  {Ё}{ {\"E} }1
  {Ж}{ {\selectfont\char198} }1
  {З}{ {\selectfont\char199} }1
  {И}{ {\selectfont\char200} }1
  {Й}{ {\selectfont\char201} }1
  {К}{ {\selectfont\char202} }1
  {Л}{ {\selectfont\char203} }1
  {М}{ {\selectfont\char204} }1
  {Н}{ {\selectfont\char205} }1
  {О}{ {\selectfont\char206} }1
  {П}{ {\selectfont\char207} }1
  {Р}{ {\selectfont\char208} }1
  {С}{ {\selectfont\char209} }1
  {Т}{ {\selectfont\char210} }1
  {У}{ {\selectfont\char211} }1
  {Ф}{ {\selectfont\char212} }1
  {Х}{ {\selectfont\char213} }1
  {Ц}{ {\selectfont\char214} }1
  {Ч}{ {\selectfont\char215} }1
  {Ш}{ {\selectfont\char216} }1
  {Щ}{ {\selectfont\char217} }1
  {Ъ}{ {\selectfont\char218} }1
  {Ы}{ {\selectfont\char219} }1
  {Ь}{ {\selectfont\char220} }1
  {Э}{ {\selectfont\char221} }1
  {Ю}{ {\selectfont\char222} }1
  {Я}{ {\selectfont\char223} }1
  {\{}{ { {\color{brackets}\{} } }1 % Цвет скобок {
  {\} }{ { {\color{brackets}\} } } }1 % Цвет скобок }
}

% НАЧАЛО ТИТУЛЬНОГО ЛИСТА
\begin{center}
\hfill \break
\hfill \break
\normalsize{МИНИСТЕРСТВО НАУКИ И ВЫСШЕГО ОБРАЗОВАНИЯ РОССИЙСКОЙ ФЕДЕРАЦИИ\\
 федеральное государственное автономное образовательное учреждение высшего образования «Санкт-Петербургский политехнический университет Петра Великого»\\[10pt]}
\normalsize{Институт компьютерных наук и кибербезопасности}\\[10pt] 
\normalsize{Высшая школа технологий искусственного интеллекта}\\[10pt] 
\normalsize{Направление: 02.03.01 Математика и компьютерные науки}\\

\hfill \break
\hfill \break
\hfill \break
\large{Проектирование WEB приложений}\\
\large{Отчёт по курсовой работе на тему:}\\
\large{\textbf{«Электронный дневник»\\}}

\hfill \break
\hfill \break
\end{center}
 
\small{ 
\begin{tabular}{lrrl}
\!\!\!Студент, & \hspace{2cm} & & \\
\!\!\!группы 5130201/20102 & \hspace{2cm} & \underline{\hspace{3cm}} & Гаар В.С. \\\\
\!\!\!Преподаватель, \hspace{2cm} & & \\
\!\!\!к.т.н., доц. & \hspace{2cm} & \underline{\hspace{3cm}} &  Попов С.Г. \\\\
&&\hspace{5cm}
\end{tabular}
\begin{flushright}
<<\underline{\hspace{1cm}}>>\underline{\hspace{2.5cm}} 2025 г.
\end{flushright}
}

\hfill \break
\hfill \break
\begin{center} \small{Санкт-Петербург, 2025} \end{center}
\thispagestyle{empty} % выключаем отображение номера для этой страницы

% КОНЕЦ ТИТУЛЬНОГО ЛИСТА
\newpage

\tableofcontents

\newpage

\cleardoublepage
\phantomsection

\addcontentsline{toc}{section}{Введение}
\section*{Введение}
В данном отчете представлено выполнение лабораторной работы по дисциплине <<Проектирование WEB приложений>>. В качестве предметной области был выбран процесс использования электронного дневнивка в школах.
В ходе работы требуется формализовать выбранную предметную область при помощи:
\begin{enumerate}
    \item текстового описания;
    \item ER-диаграмм;
    \item Use case-диаграмм и описаний.
\end{enumerate}

\newpage
\section{Аналитика}
\subsection{Описание предметной области}

Рассматриваемая предметная область -- использование электронного дневника в школах. Электронный дневник представляет собой цифровую систему, предназначенную для автоматизации процессов ведения школьной документации, взаимодействия между учениками, родителями и учителями, а также для организации учебного процесса. Основная цель системы -- облегчить доступ к информации об успеваемости, расписании, домашних заданиях и других аспектах школьной жизни. Кроме того, электронный дневник способствует повышению прозрачности образовательного процесса и упрощает администрирование учебной деятельности.

\subsubsection{Основные сущности предметной области}

\textbf{Ученик} -- человек, получающий образование в данной школе. У каждого ученика имеется персональный профиль в системе электронного дневника, где фиксируются его успеваемость, посещаемость и домашние задания. Ученик может просматривать свою статистику и получать уведомления о предстоящих контрольных и изменениях в расписании. Кроме того, ученик записывает домашние задания, которые затем проверяются учителем. Посещая уроки, ученик получает оценки за свои работы, что фиксируется в системе.

\textbf{Родитель} -- законный опекун ученика. У одного родителя может быть несколько детей-учеников. Родители имеют возможность контролировать успеваемость ребёнка, анализируя полученные оценки, пропуски занятий и выполненные домашние задания. Взаимодействие с учителями осуществляется через систему сообщений, в которой родители могут получать уведомления о поведении ребёнка, его успехах или замечаниях со стороны преподавателя.

\textbf{Класс} -- группа учеников одной ступени обучения, занимающихся по единой образовательной программе. Номер класса формируется из ступени обучения (цифра 1--11) и дополнительного идентификатора (буква А--Я). Класс является основной единицей организации учебного процесса. Каждый класс имеет своё расписание занятий, сформированное администрацией школы. В рамках класса ведётся журнал успеваемости и посещаемости, в котором фиксируются все оценки и замечания.

\textbf{Предмет} -- учебный курс, который изучается учениками в образовательном учреждении. Примерами предметов являются математика, литература, физика и т. д. Каждый предмет преподаётся в рамках определённых уроков, во время которых учитель излагает материал, задаёт домашние задания и контролирует знания учеников. В системе электронного дневника каждому предмету соответствует список домашних заданий, контрольных работ и оценок.

\textbf{Учитель} -- человек, который преподаёт в школе предметы. Учитель назначается для проведения уроков в рамках учебного расписания и может курировать один из классов как классный руководитель. Он отвечает за ведение журнала, выставление оценок, контроль посещаемости и внесение комментариев по успеваемости учеников. Кроме того, учителя могут взаимодействовать с родителями через систему сообщений, предоставляя рекомендации и замечания.

\textbf{Сообщение} -- текстовое уведомление, используемое для взаимодействия между родителями, учениками и учителями. Сообщения могут включать замечания о поведении, уведомления о предстоящих мероприятиях, просьбы и обратную связь. В системе может быть реализован функционал групповых уведомлений и рассылок. Каждое сообщение привязывается к конкретному уроку и может содержать рекомендации для учеников и родителей.

\textbf{Урок} -- учебное занятие, включающее определённый предмет и временной интервал. К уроку может быть прикреплено сообщение, учебные материалы и домашнее задание. Учитель фиксирует тему занятия, отмечает присутствующих учеников, выставляет оценки и добавляет комментарии. Уроки проводятся по установленному расписанию, а их результаты записываются в журнал.

\textbf{Расписание} -- последовательность уроков для определённого класса, составляемая на каждый учебный день. Расписание доступно для просмотра ученикам, родителям и учителям. Оно включает информацию о преподавателях, предмете и времени начала занятий. В системе предусмотрена возможность оперативного обновления расписания при отмене или переносе уроков.

\textbf{Оценка} -- балл, выставленный за выполнение учеником учебной работы. Используется пятибалльная система:
\begin{itemize}
\item 2 -- неудовлетворительно,
\item 3 -- удовлетворительно,
\item 4 -- хорошо,
\item 5 -- отлично.
\end{itemize}
Оценка может иметь вес, определяющий её значимость. Оценки выставляются учителями и отображаются в профилях учеников. Кроме того, система может поддерживать средний балл и динамику изменений успеваемости.

\textbf{Домашнее задание} -- задание, выдаваемое ученикам для самостоятельного выполнения. Ученики записывают домашние задания, а учителя могут оценивать их выполнение. Домашние задания могут содержать ссылки на дополнительные материалы и задания с автоматической проверкой. Каждое домашнее задание связано с конкретным предметом и конкретным уроком, а по его выполнению ученик получает оценку.

\textbf{Администрация школы} -- группа сотрудников, ответственных за управление учебным процессом и функционирование системы электронного дневника. Администрация обеспечивает корректную работу системы, контролирует соблюдение образовательных стандартов и организует взаимодействие между всеми участниками учебного процесса.

\subsubsection{Основные процессы}
\noindent \textbf{Процесс ведения успеваемости}
\begin{enumerate}
  \item Учитель проводит урок и оценивает знания учеников.
  \item Оценки фиксируются в электронном дневнике.
  \item Ученики и родители могут просматривать оценки и получать уведомления о новых отметках.
  \item Учитель может добавлять комментарии к оценкам.
  \item Родители имеют возможность оставлять запросы на разъяснение выставленных оценок.
\end{enumerate}

\noindent \textbf{Процесс ведения расписания}
\begin{enumerate}
  \item Администрация школы формирует расписание для каждого класса.
  \item Расписание загружается в систему электронного дневника.
  \item Учителя и ученики могут просматривать расписание и получать уведомления об изменениях.
  \item В случае изменений администрация оперативно вносит корректировки.
\end{enumerate}

\noindent \textbf{Процесс выполнения домашних заданий}
\begin{enumerate}
  \item Учитель задаёт домашнее задание в системе.
  \item Ученик выполняет задание и загружает его в электронный дневник.
  \item Учитель проверяет работу и выставляет оценку.
  \item Ученик и родители могут просмотреть оценку и комментарии к выполненной работе.
\end{enumerate}

\noindent \textbf{Процесс контроля посещаемости}
\begin{enumerate}
  \item Учитель фиксирует присутствие учеников на уроке.
  \item В случае пропуска указывается причина отсутствия.
  \item Родители получают уведомление о пропущенных занятиях.
  \item Администрация школы анализирует посещаемость учеников.
\end{enumerate}

\noindent \textbf{Формирование отчётности}
\begin{enumerate}
  \item Учителя и администрация школы формируют отчёты об успеваемости и посещаемости учеников.
  \item Родители могут скачивать отчёты о результатах своего ребёнка.
  \item Отчёты могут быть экспортированы в различные форматы (PDF, Excel) для анализа и хранения.
\end{enumerate}


\subsubsection{Вспомогательные процессы}
\noindent \textbf{Регистрация и управление профилями}
\begin{enumerate}
  \item Ученик, родитель или учитель регистрируется в системе электронного дневника.
  \item Администрация школы подтверждает регистрацию.
  \item Участник получает доступ к своим данным.
  \item В случае необходимости пользователи могут изменять свои данные и пароли.
\end{enumerate}

\noindent \textbf{Обратная связь и взаимодействие}
\begin{enumerate}
  \item Учителя, родители и ученики могут обмениваться сообщениями в системе.
  \item Сообщения могут включать текст, файлы и мультимедийные материалы.
  \item Учителя могут отправлять уведомления о важной информации.
  \item Родители могут оставлять запросы на разъяснение оценок и дисциплинарных мер.
\end{enumerate}


\noindent \textbf{Настройки и администрирование системы}
\begin{enumerate}
  \item Администраторы школы могут управлять правами доступа пользователей.
  \item Учителя могут добавлять методические материалы и учебные планы.
  \item Система может интегрироваться с другими образовательными платформами.
  \item Обновления и изменения в системе регулярно вносятся для улучшения функционала.
\end{enumerate}
  
Таким образом, электронный дневник играет важную роль в организации учебного процесса, облегчая взаимодействие между всеми участниками образовательного процесса и обеспечивая удобный доступ к важной информации.

\section{ER-диаграмма}
ER-диаграмма представлена на \hyperlink{diag:erd}{рисунке №1}.
% \newpage
% \hypertarget{diag:erd}{}
% \includepdf[pages=1, fitpaper]{./Листы/ERD.pdf}
% \newpage

\subsection{Чтение ER-диаграммы}
Родители опекают своих детей-учеников, а также реагируют на сообщения учителей.

В свою очередь, ученики состоят в классе, записывают домашние задания, посещают уроки, получают оценки. Также ученики могут просматривать актуальное расписание своего класса.

Учителя преподают предметы, задают домашние задания, ведут уроки, выставляют оценки, оставляют сообщения в качестве замечаний, а также некоторые из учителей могут руководить одним классом.

Каждое домашнее задание выдаётся по одному из предметов и задано на один конкретный урок. За каждое домашнее задание выставляется ровно одна оценка.

Каждое расписание состоит из нескольких уроков. Для одного класса может быть составлено несколько расписаний.

Каждое из сообщений прикреплено ровно к одному уроку.


\newpage
\section*{Заключение}
\addcontentsline{toc}{section}{Заключение}
Полученные знания могут быть и будут использованы в работе над последующими проектами и заданиями.

\cleardoublepage
\phantomsection
\newpage
%Список источников
\begin{thebibliography}{0}
	% \bibitem{bib:mysqldoc}
	% MySQL Documentation [Электронный ресурс] URL: https://dev.mysql.com/doc/ (дата обращения 30.04.2024).
\end{thebibliography}
\addcontentsline{toc}{section}{Список источников}
\end{document}